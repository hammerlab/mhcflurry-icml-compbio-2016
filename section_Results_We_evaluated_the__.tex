\section{Results}
We evaluated the effect of imputation, using only the training set, by down-sampling the three alleles with the most training data to a range of simulated training set sizes and testing on the remaining data (Figure ~\ref{fig:imputecomparison}). We find that imputation gives a modest improvement up to approximately 100 training samples. With more training data there is no benefit to imputation. The results are similar for the two other performance metrics (not shown).

% The figure doesn't show up in the preview but does show up if you export to pdf.
\begin{figure}[hb]
\includegraphics{figures/impute_comparison.pdf}
\caption{\label{fig:imputecomparison} MHCflurry performance on down-sampled training data with and without imputation}
\end{figure}

We compared the performance of two MHCFlurry-based models, ``mhcflurry ensemble'' and ``mhcflurry single,'' against netMHC, netMHCpan, and smmpmbec on the blind test data. The ``mhcflurry single'' model is one predictor with the architecture described previously. The MHCflurry ensemble model contains 10 predictors initialized with different random weights.

\begin{table}[hr]
\centering
\begin{tabular}{llll}
\toprule
{} &               AUC &       $F_1$ score &  Kendall's $\tau$ \\
\midrule
MHCflurry (ensemble)        &  \textbf{0.93260} &           0.78459 &   \textbf{0.58686} \\
MHCflurry (single predictor)    &           0.93225 &           0.78106 &           0.58572 \\
NetMHC                          &           0.93234 &  \textbf{0.80722} &   \textbf{0.58633} \\
NetMHCpan                       &  \textbf{0.93264} &           0.79957 &           0.58138 \\
SMM-PMBEC                       &           0.92134 &           0.79026 &           0.56488 \\
\bottomrule
\end{tabular}

\caption{Performance on BLIND}
\label{tab:measurementweighted}
\end{table}

\section{Discussion}
Imputing training data shows promise in cross-validation as a way to improve performance on alleles with few observations, but only seems to help for very small training sizes ($\leq 200$). Unfortunately, only one allele in the BLIND dataset had fewer than 200 samples. Thus, additional work is required to assess the accuracy of MHCflurry and other predictors on alleles with very few training examples. Additionally, we need to further investigate the interaction between imputation parameters, the schedule according to which the weights of imputed samples are decayed, and stopping criteria for training individual allele-specific predictors. Nonetheless, even in its preliminary state, MHCflurry 

% These are generated in the 'paper plots' notebook; do not edit by hand.
